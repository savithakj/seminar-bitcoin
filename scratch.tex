@article{lowenthal2011bitcoin,

 title={},

 author={},

 journal={},

 volume={14},

 year={2011}

}


\chapter{Related Works}
\textbf{Electronic funds transfer (EFT)} is the electronic exchange, transfer of money from one account to
another, either within a single financial institution or across multiple institutions, through computer-based 
systems.
It consists of different concepts;
\begin{itemize}
	\item Cardholder-initiated transactions, using a payment card such as a credit or debit card
	\item Direct deposit payment initiated by the payer
	\item Direct debit payments, sometimes called electronic checks, for which a business debits the consumer's bank accounts for payment for goods or services
	\item Electronic bill payment in online banking, which may be delivered by EFT or paper check
	\item Transactions involving stored value of electronic money, possibly in a private currency
	\item Electronic Benefit Transfer
\end{itemize}
EFTs include direct-debit transactions, wire transfers, direct deposits, ATM withdrawals and online 
bill pay services. Transactions are processed through the Automated Clearing House (ACH) network, 
the secure transfer system of the Federal Reserve that connects all U.S. banks, credit unions and other 
financial institutions.

For example, when you use your debit card to make a purchase at a store or online, the transaction 
is processed using an EFT system. The transaction is very similar to an ATM withdrawal, with near-
instantaneous payment to the merchant and deduction from your checking account.

Direct deposit is another form of an electronic funds transfer. In this case, funds from your employer’s
bank account are transferred electronically to your bank account, with no need for paper-based payment 
systems.

The increased use of EFTs for online bill payments, purchases and pay processes is leading to a paper-
free banking system, where a large number of invoices and payments take place over digital networks. 
EFT systems play a large role in this future, with fast, secure transactions guaranteeing a seamless transfer 
of funds within institutions or across banking networks.

EFT transactions, also known as an online transaction or PIN-debit transaction, also offer an alternative 
to signature debit transactions, which take place through one of the major credit card processing systems, 
such as Visa, MasterCard or Discover, and can cost as much as 3\% of the total purchase price. EFT 
processing, on the other hand, only charges an average of 1\% for debit card transactions.

\chapter{Existing System}
The first crypto currency to begin trading was Bitcoin in 2009. Since then, numerous crypto 
currencies have been created. Fundamentally, crypto currencies are specifications regarding the 
use of currency which seek to incorporate principles of cryptography to implement a distributed, 
decentralized and secure information economy. When comparing crypto currencies to fiat 
money, the most notable difference is in how no group or individual may accelerate, stunt or in 
any other way significantly abuse the production of money. Instead, only a certain amount of 
crypto currency is produced by the entire crypto currency system collectively, at a rate which is 
bounded by a value both prior defined and publicly known.

In centralized economic systems such as the Federal Reserve System governments regulate 
the value of currency by simply printing units of fiat money or demanding additions to digital 
banking ledgers, however governments cannot produce units of crypto currency and as such 
governments cannot provide backing for firms, banks or corporate entities which hold asset 
value measured in a decentralized crypto currency. The underlying technical system upon 
which all crypto currencies are now based was created by the anonymous group or individual 
known as Satoshi Nakamoto for the purpose of creating an economy within which the practice 
of fractional reserve banking would be fundamentally impossible.

Hundreds of crypto currency specifications now exist, most are similar to and derived from the 
first fully implemented crypto currency protocol, Bitcoin. Within crypto currency systems,
the safety, integrity, and balance of all ledgers is ensured by a swarm of mutually distrustful
parties, referred to as miners, who are, for the most part, general members of the public, actively
protecting the network by maintaining a high hash-rate difficulty for their chance at receiving
a randomly distributed small fee. Subverting the underlying security of a crypto currency is 
mathematically possible, but the cost may be unfeasibly high.

Most crypto currencies are designed to gradually introduce new units of currency, placing an 
ultimate cap on the total amount of currency that will ever be in circulation. This is done both to 
mimic the scarcity (and value) of precious metals and to avoid hyperinflation. As a result, such 
crypto currencies tend to experience hyper deflation as they grow in popularity and the amount 
of the currency in circulation approaches this finite cap. Compared with ordinary currencies 
held by financial institutions or kept as cash on hand, crypto currencies are less susceptible 
to seizure by law enforcement. Existing crypto currencies are all pseudonymous, though 
additions such as Zerocoin and its distributed laundry feature have been suggested, which would 
allow for anonymity.

\chapter{Proposed System}
Bitcoin is a form of digital currency, created and held electronically. No one controls it. Bitcoin aren’t
printed, like dollars or euros – they’re produced by lots of people running computers all around the world,
using software that solves mathematical problems. It’s the first example of a growing category of money 
known as cryptocurrency.

Bitcoin can be used to buy things electronically. In that sense, it’s like conventional dollars, euros, or yen, 
which are also traded digitally.

However, bit coin’s most important characteristic, and the thing that makes it different to conventional 
money, is that it is decentralized. No single institution controls the Bitcoin network. This puts some 
people at ease, because it means that a large bank can’t control their money.

This currency isn’t physically printed in the shadows by a central bank, unaccountable to the population, 
and making its own rules. Those banks can simply produce more money to cover the national debt, thus 
devaluing their currency.

Instead, Bitcoin is created digitally, by a community of people that anyone can join. Bitcoin are ‘mined’, 
using computing power in a distributed network. This network also processes transactions made with the 
virtual currency, effectively making Bitcoin its own payment network.

Bitcoin has several important features that set it apart from normal fiat currencies.

\begin{enumerate}
	\item It’s decentralized
	
	The Bitcoin network isn’t controlled by one central authority. Every machine that mines bitcoin and
	processes transactions makes up a part of the network, and the machines work together. That means that, 
	in theory, one central authority can’t tinker with monetary policy and cause a meltdown – or simply 
	decide to take people’s Bitcoin away from them, as the Central European Bank decided to do in Cyprus in 
	early 2013. And if some part of the network goes offline for some reason, the money keeps on flowing.
	\item It's easy to setup

	Conventional banks make you jump through hoops simply to open a bank account. Setting up merchant
	accounts for payment is another Kafkaesque task, beset by bureaucracy. However, you can set up a
	Bitcoin address in seconds, no questions asked, and with no fees payable.

	\item It's anonymous

	Well, kind of. Users can hold multiple Bitcoin addresses, and they aren’t linked to names, addresses, or
	other personally identifying information.

	\item It's completely transparent

	Bitcoin stores details of every single transaction that ever happened in the network in a huge version 
	of a general ledger, called the block chain. The block chain tells all. If you have a publicly used bitcoin
	address, anyone can tell how many Bitcoin are stored at that address. They just don’t know that it’s yours.
	There are measures that people can take to make their activities more opaque on the bitcoin network,
	though, such as not using the same Bitcoin addresses consistently, and not transferring lots of bitcoin to a
	single address.

	\item Transaction fees are miniscule

	Your bank may charge you a £10 fee for international transfers. Bitcoin doesn’t.

	\item It's fast

	You can send money anywhere and it will arrive minutes later, as soon as the bitcoin network processes
	the payment.

	\item It's non-repudiable

	When your bitcoins are sent, there’s no getting them back, unless the recipient returns them to you.
	They’re gone forever.
\end{enumerate}

%
